% -*-LaTeX-*-
% merger.tex - explain algorithm used in the near-online merger

\documentclass{article}
\usepackage[a4paper]{geometry}  % squeeze lots onto the page

\usepackage{alltt}

\begin{document}
\title{The Merger Algorithm}
\author{Keith Wansbrough\and Steven Bishop}
\date{\today}
\maketitle

We have a number of logs, each of which generates a stream of events
paired with timestamps.  Furthermore, each log has a fixed offset that
must be applied to timestamps on that log, to correct for inaccuracies
in the generating clock and for propagation delay.  The aim is to make
all the corrected timestamps use the same timebase.

Clearly we maintain a queue for each log -- this may need to be a
priority queue in (corrected) timestamp order if events can be
reordered between source and merger, or it may not.  We may apply the
correction to the timestamps immediately on receipt, and thereafter
use corrected timestamps exclusively.

The merger must generate a stream of events in chronological order.
Easy enough, except that we want it to be reasonably live, even when
one of the logs is slow or even silent.

So: how to decide when we can emit the earliest event we have, even
when one of the log queues is empty?

Imagine we record the local time at which each event arrives, as well
as its corrected timestamp.  We must make an assumption about the maximum difference in propagation
delay: \emph{If event~$A$ occurred before event~$E$, then event~$A$
cannot arrive at the merger more than $\Delta$ seconds after
event~$E$ arrived.}  Note that $\Delta$ is a global constant.

Now when we wish to emit an event, we may determine the event~$E$
that, of all events on all queues, has the earliest (corrected)
timestamp.  If we have an empty queue~$q$ at time~$t$, but $t$ is more
than $\Delta$ seconds after event~$E$ arrived, then by the assumption
it is impossible that an event on~$q$ will ever precede~$E$.  Thus
event~$E$ may safely be emitted.

More precisely:
\begin{itemize}
\item Calculate the earliest event (by event timestamp) of all events on all
queues.  Call it~$E$.  We now test to see if it is safe to emit~$E$.
\item At current time ~$t$, test whether
$t-\mathit{arrival}(E)>\Delta$.  (Note that it is safe to use the
current time here; once the condition holds, it cannot cease to hold.)
\item If the condition holds then proceed: emit event~$E$.
\item Otherwise, wait until either $\mathit{currenttime}-\mathit{arrival}(E)\geq\Delta$,
i.e., $\mathit{currenttime} \geq \mathit{arrival}(E)+\Delta$, or
until another event arrives at the head of a queue (if using priority
queues, this could be on any queue; otherwise, this could only be on
an empty queue; in either case, it may change $E$).
\item If the delay has been waited out, $E$ may be emitted; if an
event has arrived, start from the beginning, redetermining $E$.
\end{itemize}

This is pretty much the implementation of the ``pull next event off
the queues and emit it'' loop.  The only detail needed to be filled in
is the actual implementation of the wait.  A reasonable way to do it
is as follows: have a condition variable for the condition.  Signal it
when an event arrives on the head of a queue; also, immediately before
the wait, spawn a thread which simply sleeps for the desired interval,
signals the condition, and dies.  This is annoying, though, because
we'll get spurious wakeups from old threads.  Maybe we can code around
this.  Recall, though, that we'll get spurious wakeups anyway: either
we accept this, and just make sure that the algorithm above is fairly
cheap; or we arrange a boolean in conjunction with the condition
variable, that is set when there really is an event on the head of the
queue.

Or try this:
\newpage
\begin{verbatim}
timed_wait cond mut d =
    let c = Condition.create () in
    let m = Mutex.create () in
    let v = ref None in
    let done_wait_time () = Mutex.lock m; v := Some(false); Mutex.unlock m; Condition.signal c in
    let done_wait_cond () = Mutex.lock m; v := Some(true); Mutex.unlock m; Condition.signal c in
    Thread.create (function d -> Thread.delay d; done_wait_time ()) d;
    Thread.create (function cond -> Condition.wait cond mut; done_wait_cond ()) cond;
    Mutex.lock m;
    while !v = None do
      Condition.wait c m
    done;
    Mutex.unlock m;
    match !v with
      Some(b) -> b
    | None -> raise exception_cant_get_here;;
\end{verbatim}

\end{document}

