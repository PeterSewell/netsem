% -*-LaTeX-*-
% tcptest.tex  --  documentation for TCP specification testing code
%
\documentclass{article}

% magic to capture RCSID:
{\catcode`\$=9\gdef\rcsid{%
$Id: tcptest.tex,v 1.11 2003/09/30 16:19:29 smb50 Exp $}}

% decent page size:
\usepackage{geometry}

% decent fonts:
\usepackage[latin1]{inputenc}
\usepackage{ae}              % fake most characters of the EC Type1 fonts with CM Type1 fonts
\usepackage{aecompl}         % fill in the rest with bitmapped fonts
  % (note that this replacement is necessary to get good results from pdflatex)
\usepackage{charter}
\usepackage[T1]{fontenc}

% useful packages:
\usepackage{url}              % \url{...} gives URLs etc with ~ and breaks
\def\UrlFont{\normalfont\tt}  % get rid of warnings about no tt/b/n
\usepackage[dvips]{graphicx}         % graphics, PostScript scaling etc
%\usepackage[all,dvips]{xy}
\usepackage{alltt}
%\usepackage{keith}
\usepackage{multicol}

% ``futures'' or ``to-do'' environment:
\newenvironment{futures}{%
\par\textbf{Futures:}%
  \begin{itemize}%
}{%
  \end{itemize}%
}

% standard abbreviations: \cf, \qv, \ie, \eg, \vs.
% KSW 1998-08-13..2002-04-05
%
\makeatletter
\newcommand{\l@tintext}[1]{\textit{#1}}
\newcommand{\cf}{\l@tintext{cf.\@}}
\newcommand{\eg}{\l@tintext{e.g.\@}}
\newcommand{\ie}{\l@tintext{i.e.\@}}
\newcommand{\qv}{\l@tintext{q.v.\@}}
\newcommand{\vs}{\l@tintext{vs.\@}}
\newcommand{\etc}{\l@tintext{etc.\@}}
\newcommand{\adhoc}{\l@tintext{ad~hoc}}
\newcommand{\posthoc}{\l@tintext{post~hoc}}
\newcommand{\aka}{\l@tintext{a.k.a.\@}}
\newcommand{\ia}{\l@tintext{inter alia}}
\newcommand{\sic}{\l@tintext{sic}}
\newcommand{\ala}{\l@tintext{�~la}}
\newcommand{\sci}{\l@tintext{sc.\@}}
\newcommand{\etal}{\l@tintext{et\ al.\@}}
\newcommand{\passim}{\l@tintext{passim}}
\newcommand{\vv}{\l@tintext{vice versa}}
\newcommand{\exnihilo}{\l@tintext{ex nihilo}}
\newcommand{\pace}{\l@tintext{pace}}
\newcommand{\mutmut}{\l@tintext{mutatis mutandis}}
\newcommand{\perse}{\l@tintext{per se}}
\newcommand{\viz}{\l@tintext{viz.\@}}
\newcommand{\defacto}{\l@tintext{de facto}}
\newcommand{\ca}{\l@tintext{ca.\@}}
\makeatother

% document:

\begin{document}

\title{The TCP Specification Test Suite\\
       \footnotesize\normalfont\itshape Documentation: \$\rcsid\$}
\author{The Netsem Project\\[1ex]
        \normalsize Computer Laboratory\\
        \normalsize University of Cambridge\\
        \normalsize Cambridge CB3~0FD, U.K.}
\date{\today}

\maketitle

\setlength{\columnseprule}{0.2pt}
\raggedcolumns
\begin{multicols}{2}

In order to test our TCP specification we have built a suite of
testing tools.  This document explains what each tool does, and how
they fit together.

All of the tools live in the \url{Net/TCP/Test/} directory tree.  We
proceed by subdirectory.

Most of the tools were written by Steve Bishop.  They compile and run
successfully on Linux (2.4 series kernels), FreeBSD (4.6-RELEASE),
and Windows (XP Professional 2002 Version, Service Pack 1 [standard
Lab install CD]) except where noted.

A diagram showing the relationship between the key tools described in
this document can be found on the final page.

{
\tableofcontents
}

% ----------------------------------------------------------------------

\section{Trying it out}

The tools can be built by invoking \texttt{make} or for WinXP,
\texttt{nmake /f makefile.win}, in the toplevel directory.  This
invokes \texttt{make} recursively in all subdirectories.  Targets
\texttt{all} and \texttt{clean} work as expected.

A typical simple demonstration involves running several copies of
\texttt{nslogger}, listening locally.  Then a \texttt{libd}, a
\texttt{sock\_client}, and a \texttt{slurper} are arranged to connect
to them.  Next the \texttt{sock\_client} is used to talk to the
\url{libd} and issue socket calls---the results are observed
by the other running tools.

% ----------------------------------------------------------------------

\section{Implementation notes}

The testing tools are mainly implemented in the Ocaml language using C
as and when necessary.  The tools have been extensively tested with
Ocaml version 3.06, the current version at the time of writing, and
rely on both the byte-code and native-code compilers.

The \texttt{Makefiles} are designed to work with gnumake and are
incompatible with the version of make shipped as standard with
FreeBSD.  The make files, \texttt{Makefile.win} are ones specifically
written to be compatible with \texttt{nmake}, the Microsoft
Maintenance Utility.  These should be used to build the tools under
Windows.


% ----------------------------------------------------------------------

\section{Instrumented sockets library \protect\url{(nssock/)}}

This C library provides the system calls of the standard sockets
interface, prefixed by \texttt{ns\_}.  The calls have the same
behaviour as the usual ones (and indeed are implemented in terms of
them), but in addition provide instrumentation, dumping calls and
returns in HOL format to a listening socket.

To use it, the programmer rewrites the program under test to use the
renamed socket calls.  She includes \url{ns_sockets.h}, and links
against \url{libnssock.a}.  Before running her program, she defines
two environment variables, \texttt{NS\_ADDR} and \texttt{NS\_PORT}, to
the IP address (in dotted decimal format for \texttt{inet\_addr}) and
the port (in decimal for \texttt{atoi}) respectively.  A third
environment variable, \texttt{NS\_DEBUG}, may also be set to enable
TCP debugging on BSD.  The standard C programming approach is taken
here: a non-zero value represents true and will enable
\texttt{TCP\_DEBUG}.

When the program first invokes a socket function, the internal
function \texttt{ns\_init} is invoked, which opens a socket connection
to the address and port specified.  All socket calls will write a
HOL-format rendering of the call to this socket, and a HOL-format
rendering of its return (if any).  When the program terminates, the
socket connection is implicitly closed -- no explicit closing of the
connection is performed.  It is worth noting here that if
\texttt{TCP\_DEBUG} is to be enabled, it is not enabled on the logging
socket created by \texttt{ns\_init}.  Instead it is enabled on all
sockets created through the \texttt{ns\_socket} call.  This makes sense:
we are instrumenting the sockets library and do not want our
instrumentation to interfere.

Timestamps are affixed to each label as a comment before it, in the
form of a decimal number of seconds since the epoch, with six decimal
places after the point.  No correction is applied; this is the system
time (in UTC).  The precise time is the moment before the system call,
and the moment after the return. Obtaining an accurate system
timestamp on Windows XP is impossible. Keith wrote some code to work
around this (see \url{nssock/tsctime} and
\url{http://www.cl.cam.ac.uk/users/kw217/useful/win32time.html} for
more information).

\texttt{ns\_init} attempts to use a high-numbered file descriptor to make
its connection, so as not to interfere with \texttt{fd} numbers used
by the main program.  This does not make sense on Windows, so no
attempt is made on this platform.

The \texttt{ns\_*} functions invoke \texttt{prn\_call} and
\texttt{prn\_ret} to do their dirty work; they're varargs printf-y
functions that output a HOL-style transition label.

\texttt{README} documents the differences between platforms, some
rationale and problems, and is worth reading.

\begin{futures}
\item It should be possible to use a Unix domain socket instead of an
Internet domain socket.
\item There should be some attempt at an explicit close when a process
dies.  This would be particularly easy on Windows, where the program
probably calls \texttt{WSACleanUp()}.
\end{futures}

% ----------------------------------------------------------------------

\section{Common library support \protect\url{(common/)}}

Many of the testing tools rely on common components, \eg\  parsers,
type definitions and pretty printers, which can be found here.  All
components are built using both the byte-code and native-code Ocaml
compilers producing \texttt{.cmo} and \texttt{.cmx} object files respectively.

\begin{itemize}
\item \texttt{nettypes} and \texttt{netconv} define our base types and
operations upon them;
\item \texttt{holtypes} defines datatypes synonymous to those
representing datagrams in our HOL model and transformations upon them;
\item \texttt{tcpcbtypes} defines datatypes synonymous to the
representation of TCP debug messages in our model;
\item \texttt{netipreass} performs IP reassembly over real IP packets;
\item \texttt{net2hol} and \texttt{hol2net} convert from real packets
and datagrams to our HOL model representations and \vv;
\item \texttt{render} and \texttt{renderlib} provide support for
common rendering tasks, \eg\  pretty-printing of IP addresses;
\item \texttt{librender}, \texttt{holrender} and \texttt{tcpcbrender}
provide rendering to HOL model transition labels for lib, datagram and
TCP debug datatypes;
\item \texttt{lexer}, \texttt{parser}, \texttt{parserlib},
\texttt{holparselib} and \texttt{tcpcbparserlib} implement a parser
for HOL model transition labels;
\item \texttt{libcalls} defines a datatype for lib calls and their
return values and implements functions for making such calls;
\item \texttt{ntpheader} provides a pretty-printer to produce \texttt{ntp} timestamps;
\item \texttt{sock} provides useful high-level wrappers for standard
socket calls such as \texttt{send}.
\end{itemize}

% ----------------------------------------------------------------------

\section{OCaml LIB binding \protect\url{(ocamllib/)}}

This OCaml module is an OCaml binding for the \texttt{nssock} sockets
library, providing the LIB interface that is defined by our TCP
specification (see \url{Net/TCP/Spec1/TCP1_LIBinterfaceScript.sml}).

To use it, make sure OCaml can see the library \url{ocamllib.cm[x]a}
(\ie\  its path should be on the OCaml include path), and put
\texttt{open Ocamllib} in your program.  Everything should `just
work'.

NB: since \url{ocamllib} is linked against \url{nssock}, programs
using it must define the environment variables described there, \ie\
\texttt{NS\_ADDR}, \texttt{NS\_PORT} and optionally,
\texttt{NS\_DEBUG}, to describe the connection used for
instrumentation output. The programs will also have to be linked
against \url{libnssock.a}.

Note that \url{ocamllib} itself performs no logging or other
instrumentation.

\url{ocamllib.cm[x]a} is the OCaml binding, \url{ocamllib.cmi} is its
interface, \url{libocamllib[native].a} (from \url{ocamllibc.c}) is the glue
code that binds the \texttt{ns\_*} calls to OCaml, and
\url{dllocamllib[native].so} is the same but as a shared object rather than a
static library.  See \url{../libd/Makefile} for an example of how to
build something using \url{ocamllib}.

\url{libfunmappings.txt} shows the correspondence between LIB calls
and socket calls on Unix and Windows.  \emph{Note that there are still a few
unresolved issues; see the \url{README}}.

% ----------------------------------------------------------------------

\section{LIB call d�mon \protect\url{(libd/)}}

This OCaml program is a d�mon that performs LIB calls as demanded
by remotely-provided messages.

To use it, run \url{libd}.  Then (on the same or another machine)
initiate a connection to the port on which \url{libd} is
listening.  Give instructions in the form of HOL call labels
(\texttt{Lh\_call} \dots), and they will be performed by \url{libd}.

NB: since \url{libd} is linked against \url{ocamllib}, and
\url{ocamllib} against \url{nssock}, the environment variables used by
the latter, \ie\  \texttt{NS\_ADDR}, \texttt{NS\_PORT} and optionally,
\texttt{NS\_DEBUG}, must be set.  These describe the connection to be
used for instrumentation output.

Thus when \url{libd} is run, a program should be listening on this IP
and port also.  It will receive the HOL call labels (which should be
the same as the messages, but with times prefixed), along with the
HOL return labels.

The IP and port, or Unix domain socket on which \url{libd} listens,
are specified on the \url{libd} command line.

The action of the program is simply to repeatedly invoke the parser.
The parser parses each call it receives and the corresponding rule
returns an encoding of the call that is eventually executed. Return
values \emph{are} captured and returned on the command socket as well
as being reported by the instrumentation in \url{nssock}.

All errors generated by the program are reported to \url{stderr}
except those raised through the calls to \url{ocamllib}, which are reported back
to the callee via the command socket.

Note that \url{libd} uses ordinary (uninstrumented) \texttt{Unix}
socket calls to establish its communication channel, not
(instrumented) \texttt{Ocamllib} calls. This prevents interference
with the system we are trying to test and instrument.


% ----------------------------------------------------------------------

\section{OCaml extensions \protect\url{(ocamlext/)}}

These three OCaml modules provide some extensions to OCaml:
\url{Platform} provides functions specific to individual target
platforms, \url{Rawsock} provides the ability to use raw sockets with
the OCaml sockets binding, and \url{ThreadParsing} is a thread-safe
version of the Ocaml parser.

The \url{Platform} module exposes a function \texttt{check\_platform}
which returns one of \texttt{WIN32}, \texttt{LINUX}, \texttt{BSD}, or
\texttt{UNKNOWN} (constructors of the type \texttt{platform}).  It also
provides Ocaml with pseudo-tty control on linux, extra process
control functions under Win32, accurate \texttt{gettimeofday()} functions on
Unixes and an implementation of \texttt{condition\_timedwait()}.

The \url{Threadparsing} module and the accompanying script,
\url{parser_hack_script}, are used to create a thread-safe parser.
The standard Ocaml parser generated through Ocamlyacc is not
re-entrant in OCaml versions 3.06 and prior.  A parser is first
specified and created through a call to \url{ocamlyacc}.  The source
file produced by \url{ocamlyacc} is then specified as the argument to
\url{parser_hack_script} which replaces all parser module calls with
the equivalent thread-safe calls in \url{Threadparsing}.  All parsers
that are created by this script must be linked against
\url{threadparsing.cm[x]a}.  NB: each instance of the thread-safe
parser must be initialised through a call to
\texttt{create\_parse\_env} in the caller's code.

The \url{Rawsock} module is used in conjunction with the standard
OCaml module \url{Unix}.  It exposes two functions.
\texttt{raw\_socket} is like \texttt{Unix.socket}, but returns a raw
socket rather than an ``ordinary'' socket (obviously this only works
if the caller has appropriate permissions).
\texttt{raw\_sockopt\_hdrincl} sets the flag on a raw socket to say
that the user will provide their own IP header (the system should not
itself generate one).  This corresponds to the C call \texttt{setsockopt}
setting \texttt{SOL\_IP}'s (\aka{} \texttt{IPPROTO\_IP}'s)
\texttt{IP\_HDRINCL} to 1.  Note that the socket generated is usable
by all the normal socket calls in the OCaml \texttt{Unix} module.  To
achieve this seamless interoperability some internal OCaml ``unixsupport''
definitions and calls are used.

OCaml programs using \url{Rawsock} must open \texttt{Unix} before
\texttt{Rawsock}, and must link against OCAML's \texttt{unix.cm[x]a},
\texttt{rawsock.cm[x]a} and \texttt{librawsock.a}.  Using
\url{Platform} requires you to link against \texttt{platform.cm[x]a}
and \texttt{libplatform.a} as well.

A \url{README} is provided, explaining how to use the system.

Originally, we tried modifying/extending the existing OCaml library.
But in view of code ownership, future revisions of OCaml and
maintenance effort, and so on, we felt it was better to write a
separate module containing our enhancements.

These modules are used by a number of other programs in the testing
system.

% ----------------------------------------------------------------------

\section{Network sniffer \protect\url{(slurp/)}}

This OCaml program makes a socket connection, and then sniffs traffic
and outputs it in a timely manner to the connection, using the HOL
datagram format.

To use it, start up a program to listen on an appropriate address.
Then invoke \texttt{./slurp} with at least the interface to sniff on
and the address of the connection to make (either
\texttt{TCP}~\mbox{\textit{ip-addr}}~\textit{port} or
\texttt{UNIX}~\textit{socketname}).  Labels will be transmitted to the
listening socket (nothing goes in the reverse direction).  Details of
command-line options may be obtained by invoking \texttt{./slurp} with
no arguments.

NB: since the slurper requires promiscuous access to the network, it
must be run as root, or with an appropriate capability
(\texttt{CAP\_NET\_ADMIN} on Linux) granted.

By default, all labels are receive labels.  If the option
\texttt{-h}~\mbox{\textit{host-ip-addr}} is used, packets with a
source address equal to \mbox{\textit{host-ip-addr}} will be given
send labels instead.

To restrict the captured traffic, a filter (in PCap format) may be
appended to the command line.  This will be compiled and used by PCap
in the normal way.  A very common command line, then, would be
\texttt{./slurp -h~192.168.0.12 eth0 TCP 192.168.0.2 5001
``tcp (src~192.168.0.12) or (dst~192.168.0.12)''}.

Timestamps are affixed to each label as a comment before it, in the
form of a decimal number of seconds since the epoch, with six decimal
places after the point.  No correction is applied; this is the system
time (in UTC).  The precise time is the moment the packet was captured
by \texttt{libpcap} \emph{(** is this true?)}.

\url{slurp} is designed to handle TCP and ICMP; it should bin other
kinds of traffic.

\url{slurp} does IP reassembly following what we hope is the ``right''
algorithm; in this case, the time used is the time of the \emph{last}
packet to arrive (\ie, when the last hole is filled in).

Although the code is there, \url{slurp} is not currently capable of
operating from a PCap capture file. Files:
\begin{description}
\item[slurp.ml] is the main code.

\item[pcapinterface.c, .ml, .mli] is an OCaml binding for PCap,
covering just what we need (it's not a complete binding, since that
seemed pointless and time-consuming).

\item[pcapfile] is old code of Keith's for parsing a PCap capture
file; none of this is currently used but is here in case required.

\item[debugrender] a trivial old renderer for the wire-format types in
nettypes, which might be useful one day.
\end{description}

\begin{futures}
\item Allow \url{slurp} to operate from a PCap capture file (not sure
if this would be useful for anything, but it would be cool).
\end{futures}

% ----------------------------------------------------------------------

\section{Datagram injector \protect\url{(injector/)}}

This OCaml program is a d�mon that injects datagrams into the network
as demanded by remotely-provided messages.

To use it, invoke \url{./injector}, passing either
\texttt{TCP}~\textit{ip-addr}~\textit{port} or
\texttt{UNIX}~\textit{socketname} as argument.  Now initiate a
connection to the specified socket, on which \url{injector} will be
listening.  Give instructions in the form of HOL send-datagram labels
(\texttt{Lh\_senddatagram \dots}), and the corresponding IP datagrams
will be injected into the network.

NB: since the injector requires raw-sockets access to the network, it
must be run as root, or with an appropriate capability
(\texttt{CAP\_NET\_ADMIN} on Linux) granted.

Internally, \texttt{injector} uses the OCaml parser in \url{common/}
to read its input.  It then translates the HOL-format datagram into an
IP datagram, providing a number of unspecifiable data fields in the
process.  Finally, it emits the datagram via a raw socket, using the
OCaml extensions of \url{ocamlext/}.

No response is made via the controlling socket.  In the case of a
parse error (or other error), \texttt{injector} simply falls
over with an uncaught exception.

Fragmentation is handled by the native IP stack; the HOL label may
describe a large datagram, and if necessary it will be fragmented
transparently.

The following fields are generated by \texttt{injector}, rather than
being determined by the HOL label:
\begin{description}
\item[IPID] Initialised based upon the current time at startup, and
incremented by 1 for each datagram sent.
\item[TOS] Defaults to 0x00.
\item[TTL] Defaults to 0x40.
\item[TCP checksum] Computed correctly.
\item[IP checksum] The IP checksum is computed by the native IP stack
on BSD and is simply passed as zero from \texttt{injector}.  On Linux
and Windows the IP checksum is computed by the translation code.
\item[Fragmentation] see above -- not performed although we currently
set the \texttt{Don't fragment} bit in the IP header.
\item[Reserved] Set to zero.
\item[TCP options] Content specified by HOL, but order is chosen in
\texttt{injector}, along with padding \emph{(** investigate details!  Do we
take care to align timestamp \ala{} TCPv1p253?)}.
\end{description}

% ----------------------------------------------------------------------

\section{BSD Kernel TCP debugging \protect\url{(utils/holtcpcb-v8/)}}

The BSD kernel provides debugging hooks into its TCP stack. A socket
with the option \texttt{SO\_DEBUG} set inserts debug records
containing IP and TCP headers, a timestamp and a copy of the current
TCP control block into a ring buffer at certain checkpoints in the
code. The content of this ring buffer is available to user
applications with the correct permissions, \ie\ root.

BSD provides a tool, \url{trpt}, that allows the ring buffer to be
inspected. Our tool, \url{holtcpcb-v8}, is based heavily on \url{trpt}
although was written almost entirely from scratch to remove
over-complicated code, fix out-by-one errors and provide output in the
format we require, \ie\ HOL-style trace records. The trace records
that are output are mostly an encoding of data from the recorded
TCP control block.

Unfortunately, the standard BSD 4.6 kernel does not use high-precision
timestamps for recording the events recorded in the TCP debug ring
buffer. A set of patches to fix this problem are provided and described
in \url{notes/bsd-notes/}. This tool requires a kernel that was built with
these patches applied in order to function correctly. A more detailed discussion on the
timestamp problem can be found in \url{notes/timestamps.txt}. \emph{NB: for
this tool to function the BSD kernel must be built with the
TCP\_DEBUG option set!}

\texttt{holtcpcb-v8} takes several command-line arguments. An IP
address and port can be specified for output to be sent to, otherwise
output defaults to \texttt{stdout}. The \texttt{-z} option prevents
the display of zeroed control blocks. These are mainly generated by
events such as the TCP stack sending RST packets for connections to
unbound ports, where a socket and its associated state do not and will
not exist. The \texttt{-n} \emph{actionmask} option is followed by an
action mask value constructed from the following flags: 0x01 input,
0x02 output, 0x04 drop and 0x08 user. This options ensures only output
of the requested type is produced.

Unlike \texttt{trpt} which displays everything in the ring-buffer,
\texttt{holtcpcb-v8} functions by default much like \texttt{trpt} running in
``follow'' mode, \ie\ it waits until something new is placed in the
buffer, outputs it then waiting until new input arrives. New input
is only output if the event occured on the same socket.

% ----------------------------------------------------------------------

\section{Sniffer/injector identity testing \protect\url{(identtest/)}}

The programs in this directory are intended to test \texttt{slurp} and
\texttt{injector}, verifying that $\texttt{slurp} \circ
\texttt{injector} = \textit{id}_{\textit{HOL}}$ and $\texttt{injector}
\circ \texttt{slurp} = \textit{id}_{\textit{wire}}$ (where the latter
identity is loose wrt the unspecified fields).

An internal ASCII format, \emph{numpty format}, is used to store
wire-format packets.  The format has no header, and a packet consists
of three fields, each of the form
\texttt{*}\textit{FieldName}\texttt{=}\textit{body} and beginning on a
new line.  The first field is a textual comment (running to the end of
the line) which is a human-readable description of the packet.  The
second field is the length of the packet in bytes (in ASCII decimal).
The third field is the packet data, stored as ASCII hexadecimal words
in network byte order, space-separated, starting on a new line, with 8
words (16 bytes) per line except the last, terminated by a new line
containing \texttt{*-}.  The file terminates with EOF.  Here is a
sample packet:
\begin{alltt}
*Comment=The first packet
*Length=72
*Data=
 4510 0048 1a65 4000 4006 9edc c0a8 000c
 c0a8 0002 0016 03fc 50c1 da96 c742 831a
 8018 e240 8f61 0000 0101 080a 009a a14a
 074f 0afa 0000 000b 3798 d22e 1583 5995
 c791 033d 7830 99c1
*-
\end{alltt}

\texttt{pcapslurp} is a C program that uses \texttt{libpcap} to slurp
packets from the wire and store them in {\em numpty} format.

The command line options may be obtained by typing \url{./pcapslurp}.
 The number of packets to capture must be specified, and one may
specify the interface to capture on (defaults to whatever
\texttt{libpcap} defaults to, probably the first non-loopback
interface), a pcap filter to apply to restrict the capture, and a file
to output the captures to (defaults to \texttt{stdout}).

\texttt{fileinject} is a C program that inverts \texttt{pcapslurp}: it
takes a filename, parses it by ``brute force'', and uses raw sockets
to inject it into the network.  This program is not currently used by
the identity checker but is kept around for historical and debugging
purposes.

\texttt{pcapslurp} and \texttt{fileinject} are written in C to make it
clear that the minimum possible is going on between the capture and
writing to file or between the read and the send; it obviously does
not reuse any of the code from other parts of the system, as it is
intended to cross-check them.

\texttt{lexer.mll} and \texttt{parser.mly} are an OCaml lexer and
parser for {\em numpty} format.  \texttt{errors.ml} defines an exception
type used everywhere else; it belongs in the preamble of
\texttt{parser.mly} but cannot be placed there because while
\texttt{ocamlyacc} exposes datatypes defined in the preamble in the
interface of the parser, it does not expose exceptions defined there.

\texttt{injectslurp} is an OCaml program that performs one half of the
identity testing.  It addresses the test of $\texttt{slurp} \circ
\texttt{inject} = \textit{id}_{\textit{HOL}}$.  On the command line it
accepts as arguments: the interface to sniff, the file containing the
HOL-format packets for injection, and the file to store the
re-slurped packets in.

Firstly an \texttt{injector} process is created.  A thread which
listens on a socket and writes the results to the output filename
specified on the command line is also spawned.  \texttt{slurp} is
forked and set to point at the listening socket.  The input file is
sent to the injector which outputs (injects) each packet on a raw
socket as it goes.  These packets will be seen by \texttt{slurp},
slurped, and eventually written to the file.

Finally the original input is diffed against the slurped output modulo
comments and the inadvertible \texttt{Lh\_recv/Lh\_send} difference.  Any
differences are reported to the console directly by \texttt{diff}.

\texttt{slurpinject} is another Ocaml program providing identity
testing in the other direction, \ie\  the test ($\texttt{inject} \circ
\texttt{slurp} = \textit{id}_{\textit{wire}}$).  It accepts the
following arguments: the interface to sniff, the {\em numpty-formatted} file
containing the packets to inject, and the filename where the final output
should be stored.

\texttt{slurpinject} proceeds by parsing the {\em numpty-format} input file
and injecting the packets specified there into the network.  These are
retrieved by \texttt{slurp}.  The slurped datagrams are then re-injected
by an \texttt{injector} instance and finally re-captured, in {\em numpty} format, by
\texttt{pcapslurp}.  The initial and final {\em numpty} format packets
are diffed.  Any differences are written to the output file specifed on
the command line.

In this situation, testing cannot easily be done online due to the potential
confusion between original packets and their slurped and injected
copies.  This is the motivation for only injecting previously captured
packets stored in the {\em numpty} format.

\url{testfile} is a {\em numpty-format} file containing some packets slurped
from an SSH session between \texttt{kurt} and \texttt{alan}.

\begin{futures}
\item Verify that the identity holds on WinXP.
\item Tidy up the code to use Ocaml's pattern matching library instead
of \url{sed}.
\end{futures}

% ----------------------------------------------------------------------

\section{TCP Testing Host Executive Engine \protect\url{(tthee/)}}

\texttt{TThee} is God.  It commands the test suite from on high.  It
defines an API for controlling the tools (\url{libd},
\url{holtcpcb-v8}, \url{slurp} and \url{injector}) on local or remote
test platforms.  It permits test descriptions to be written that rely
upon the other testing tools.  These descriptions are still relatively
low-level and require some effort to produce; the next chapter
describes a higher-level test description API that builds upon
\texttt{TThee}.  During a typical test run, several tools are invoked
and the results produced are merged in corrected chronological order
to produce a test-run trace.

One of the inherent problems with merging in chronological order is
that of possible time differences on test hosts.  To correct
for this \url{ntp} is used on each host to slowly synchronise
clocks.  Each test tool that produces instrumented output reports the
current \url{ntp} offset at the time it was started.  These are used by
\texttt{TThee} to correct the results from individual tools before
merging into chronological order.

\texttt{TThee} has some important features that are worth noting here:
\begin{itemize}
\item \texttt{Tool execution}: An interface is provided for executing
the previously named test tools either locally or remotely. Local
execution is achieved through Ocaml's \texttt{create\_process}
call.  Remote executions are a little more difficult.  On Linux and BSD
targets, remote execution is achieved through \url{ssh}'s remote
commmand execution feature.  On Windows---for which a suitable
opensource ssh d�mon could not be found---the same is achieved
through the use of the \url{custom_rsh} tool.  This is discussed in the later
chapter on the \protect\url{utils/} directory.

\item \texttt{Command and Control}: A selection of the previously
mentioned tools require command messages sent over a
control socket, \eg\  the injector requires datagrams to inject.  In
\texttt{Tthee} the details of this are abstracted away.  Instead there
are suitable function calls available for requesting tools to
perform certain operations.  These are exposed to the caller.

\item \texttt{Logging and the Selector}: Some tools also require
logging channels to which instrumented output can be written, \eg\
\texttt{libd}.  \texttt{Tthee} allows the creation of new log channels
to which tools being executed, either locally or remotely, can
connnect.  Each log has associated with it a selector thread.  This is
responsible for parsing the input, applying \url{ntp} time corrections
and providing the final messages to any registered callbacks.  Once an
end-of-file is read on the logging channel, its descriptor is closed,
callbacks notified and the selector thread exits.

\item \texttt{Callbacks (anybody there?)}: Callback functions can be
registered with the selector for individual log channels.  These are
called each time a selector thread parses and corrects a message
received on a logging channel.  Callbacks are user registerable.  This
facilitates an environment where a caller may wish to make future
\texttt{Tthee} requests based upon passed observed behaviour of the
system.

\item \texttt{Merging the results}: The results received from all
tools are mergered by the merging algorithm described in
\protect\url{doc/merger.ps}.  The merger receives messages from each
of the selector threads through a registered callback with each.
These messages are already \url{ntp} offset corrected but may be
subject to a systematic delay due to propogation delays or system
scheduling issues.  The merger can correct for these delays through a
caller specified correction factor.  Once messages are finally
corrected, merging proceeds in strict chronological order.  The
merging occurs on-line; this leads to further complications that are
discussed in \protect\url{doc/merger.ps}.  The final mergered output
is written to an output channel provided by the caller.
\end{itemize}

\begin{futures}
\item DO NOT BREAK IT!
\end{futures}
% ----------------------------------------------------------------------

\section{Test generation \protect\url{(tests/)}}

This directory consists of several subdirectories each of which
perform certain tests on our test network and log the results.  We
proceed by subdirectory:

\begin{description}
\item[common/] contains definitions specific to our test network,
\eg\ IP addresses for its hosts and \texttt{Tthee} execution
parameters. Many of the other tests require the definitions found here.

\item[autotest/] is a suite for automatically and exhaustively testing
TCP implementations on our test hosts.  \emph{Note: this is currently work in
progress}.  The core code currently contains functions to exhaustively test
all socket calls and most of the rules covered by our TCP specification, to
initialise and cleanup a \texttt{Tthee} testing environment and to
force a test socket into a specified bound, connected or
TCP-state-diagram state. One is also able to specify abstractly a simple test
``thunk'' describing a test to be performed using \texttt{Tthee}
primitives, along with an initial state for the socket(s).  The testing
suite performs the test on all the test hosts in our test network.
The script, \url{run}, calls the \url{autotest} program storing the
test results within our test results directory heirarchy. The related
script, \url{test}, performs the same tests leaving the result stored
in a temporary directory within the test results directory
heirarchy. This prevents development test runs from being confused
with real results that are to be checked by HOL.

\item[adhoc*/] directories contain short one-off tests written above the
\texttt{Tthee} library that check or demonstrate certain properties
that we were interested in at some previous point in time.  These often
contain a \texttt{README} file to explain the test in more detail.

\item[tthee*/] directories contain tests written above \texttt{Tthee}
to test the implementation of \texttt{Tthee} and its interaction with
different test host platform types.  The aim here was to ensure that
\texttt{Tthee} was able to successfully fork, terminate and
communicate with test tools on remote platforms.

\item[parser\_test] contains a simple application that serves to test
the parser found in \url{common/}.  It parses strings on \texttt{stdin}
and displays information on the message parsed to
\texttt{stdout}.  Incorrectly formatted messages raise an exception
that is not caught, hence terminating the program.

\item[ssh*/] directories contain the \emph{test results} from some very
early tests where we used some of our test tools to interact with a
real ssh d�mon running on one of our test network hosts.

\end{description}

% ----------------------------------------------------------------------

\section{Useful utilities \protect\url{(utils/)}}

This directory consists of useful utilities that are individually
documented below by subdirectory.

\begin{description}
\item[ns\_logger/] is a C program that binds to an address, listens and
accepts connections, dumping all received data to the conole.  It is essential
for hooking up to the above tools during debugging!

\item[socktools/] contains two C programs: \url{sockclient} and
\url{sockserv}. The \url{sockserv} tool is an ``echo'' server, \ie\  it
accepts connections and echos any received data to the console and
to the sender.  On the otherhand, \url{sockcl
ient} is a ``telnet
client without telnet control codes'' (\ie\  a raw sender/receiver)
useful for debugging purposes.  Both of these tools are written in C
and are linked against \url{libnssock}, hence requiring a logging IP and
port to be specifed in the environment variables \texttt{NS\_ADDR} and
\texttt{NS\_PORT}.  A useful logger is \url{ns_logger} described above.

\item[custom\_rsh/] is a DEPRECATED version of the tool that allows
testing tools to be started on remote Windows machines from
\texttt{Tthee}.  It is written in Ocaml and suffers from unsolved
control-flow problems.  It should not be used!

\item[custom\_rsh2/] is a working version of the previous tool written
in C.  It can be executed as a Windows service (d�mon) without a console
using the handy NT utility, \url{srvany.exe}, as a stub.  The tool
requires that three environment variables are set:
\texttt{EXEC\_LIBD}, \texttt{EXEC\_SLURP} and \texttt{EXEC\_INJECTOR}.
These should contain the full paths to the respective programs on the
Windows host. \url{custom_rsh2} listens on an IP and port specified on
its command line. Accepted connects spawn a new handling thread that
accepts and executes one command. These commands fork one of the three
test tools mentioned previously. Remotely closing this connection
forces the forked process to be terminated. Messages on the connected
socket should be of the form:
\texttt{['L'|'S'|'I'][name=string\$]*['*'string]}, \eg\
\begin{alltt}
SNS\_ADDR=192.168.0.2\$NS\_PORT=2000\$*\\slurp TCP 192.168.0.3 2000
\end{alltt}
where L, S and I specifcy \texttt{libd}, \texttt{slurper} and
\texttt{injector} respectively and the whole command is on a single
line. The example executes the slurper in the environment extended as
shown.

\item[holtcpcb/] is a DEPRECATED modification of the standard BSD
\url{trpt} program that outputs \texttt{TCP\_DEBUG} information from any
socket with \texttt{SO\_DEBUG} set.  The tool outputs HOL-style records
representing the socket's state at each debugging checkpoint.  The tool
was deprecated for two reasons:
\begin{enumerate}
\item The original \url{trpt} tool suffered from many out-by-one
errors and was too complicated to fix properly;
\item Standard BSD kernels did not timestamp \texttt{TCP\_DEBUG}
messages accurately enough for our requirments.
\end{enumerate}

\item[holtcpcb-v8/] is a newer version of the previous tool and is
described in an earlier chapter of this document.

\end{description}

% ----------------------------------------------------------------------

\section{Code distribution \protect\url{(scripts/)}}

This directory contains shell scripts that perform useful tasks.

The \url{distribute} script packages the local version of the
\url{Net/TCP/Test/} directory tree into a gzipped tarball and securely
copies it to each non-Windows machine on our test network: \url{alan},
\url{kurt}, \url{emil} and \url{john}.

Connecting via \url{ssh} it unpackages the directory on each host into
\url{/home/Net/TCP/Test} and proceeds to call \url{make} at the top
level.  Finally, for tools that require extra privileges, it changes
their ownership to  user \texttt{root} and group \texttt{wheel}, and
sets their \texttt{setuid} bit.

To make life simpler, OCaml lives in the same place on the subnet
machines as on the lab machines: \url{/usr/groups/netsem/ocaml/}.  For
the script to work without manual intervention, RSA key authentication
must be setup for \url{ssh} connections to the machines on the test
network.

The \url{packagetsc} script packages Keith's accurate Windows XP
timestamping code for independent distribution. This was written in
order to release the timestamping code to the wider community under a
BSD-style license.

All of our test traces are eventually checked by our HOL specification
of TCP. This is a compute-expensive operation. In search of efficiency
we distribute this accross several machines. The scripts in
\url{scripts/build} are used during the preparation of these
machines. See \url{notes/newmachine.txt} and \url{TCP/Spec1} for more
information.

% -----------------------------------------------------------------------

\section{BSD experimentation \protect\url{(bsd/)}}

The contents of this directory are \emph{DEPRECATED}.  Do not use!
This directory contains various proof-of-concept and experimental
code.

\begin{description}
\item[rawtcptest] C code for the first raw sockets test on BSD we ever
wrote.

\item[rawtest1] OCaml hack for using raw sockets to emit packets.

\item[simple, socktest1, socktest2] Experimentation with the OCaml
sockets binding.
\end{description}


% -----------------------------------------------------------------------

\section{WinXP experimentation\\ \protect\url{(winxp/)}}

The contents of this directory are \emph{DEPRECATED}.  Do not use!
This directory contains various proof-of-concept and experimental code.

\begin{description}
\item[rawtest] C code that implements raw socket sending on WinXP.
\end{description}

% -----------------------------------------------------------------------

\section{OCaml experimentation\\ \protect\url{(ocamlhack/)}}

The contents of this directory are \emph{DEPRECATED}.  Do not use!
This was a proof-of-concept implementation of the raw sockets stuff,
now superseded by \url{ocamlext/Rawsock}.


% -----------------------------------------------------------------------

\section{Merging experimentation \protect\url{(merge/)}}

The contents of this directory are \emph{DEPRECATED}.  Do not use!
This tool attempted merging before \texttt{Tthee} superceded.

% -----------------------------------------------------------------------

\section{Notes \protect\url{(notes/)}}

This directory contains some miscellaneous notes about things that are
of interest and turned up during development.  Also see the
\protect\url{README} files scattered about the directories.

\begin{description}
\item[netfilter]  At one stage we considered using \texttt{netfilter}
for packet slurping and injection; we eventually realised that
injection was not possible in this system.  However, the results of
our investigations are recorded here.

\item[thalamus-move]  This document lists the requirements and details
of the move of \texttt{thalamus} to FE21, and various other things to
do with the experimental setup.

\item[fds] This discusses the issues of using high-numbered file
descriptors to avoid disturbing existing code (for \url{nssock/}).
Note in particular that there is an issue with OCaml's bytecode
interpreter here.

\item[bsd-hosts/] Documentation on the installation and configuration
of our BSD test hosts (including relevant kernel source patches for
increased TCP\_DEBUG timestamp accuracy).

\item[linux-hosts/] Documentation on the installation and
configuration of our Linux test hosts.

\item[timestamps] An explanation on the TCP\_DEBUG timestamp accuracy
problem in the BSD kernel.  This is fixed by the patches available in \url{bsd-hosts/}.

\item[netsniffers] Some notes on the various commercial and
non-commercial packet sniffers available.  We thought at one time that
a packet sniffer could be turned into the slurper.

\item[auto-testing] Some comprehensive notes on what automatic test
code we require.  This is currently being implemented in \url{tests/}.

\item[spec-notes] Steve's thoughts on a version of the TCP
specification.  (Mainly things that Steve believes are not quite right or
are incomplete).

\item[newmachine.txt] A document describing how to configure a new
machine ready for use in our distributed checking of traces under HOL.

\end{description}

\section{This document \protect\url{(doc/)}}

This document lives in \url{Net/TCP/Test/doc/}.  Please keep it
up-to-date with any changes!

\end{multicols}

\begin{figure}
\centering
\includegraphics[height=15cm]{protstack.eps}
\caption{Testing Architecture} \label{f1}
\end{figure}

\end{document}

