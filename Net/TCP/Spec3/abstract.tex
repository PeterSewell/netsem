Despite more than 30 years of research on protocol specification, the
major protocols deployed in the Internet, such as TCP, are described
only in informal prose RFCs and executable code.  In part this is
because the scale and complexity of these protocols makes them
challenging targets for formal descriptions, and because techniques
for mathematically rigorous (but appropriately loose) specification
are not in common use.

In this work we show how these difficulties can be addressed.  We
develop a high-level specification for TCP and the Sockets API,
describing the byte-stream service that TCP provides to users,
expressed in the formalised mathematics of the HOL proof assistant.
This complements our previous low-level specification of the protocol
internals, and makes it possible for the first time to state what it
means for TCP to be correct: that the protocol implements the service.
We define a precise abstraction function between the models and
validate it by testing, using verified testing infrastructure within
HOL. Some errors may remain, of course, especially as our resources
for testing were limited, but it would be straightforward to use the
method on a larger scale.  This is a pragmatic alternative to full
proof, providing reasonable confidence at a relatively low entry cost.

Together with our previous validation of the low-level model, this
shows how one can rigorously tie together concrete implementations,
low-level protocol models, and specifications of the services they
claim to provide, dealing with the complexity of real-world protocols
throughout.

Similar techniques should be applicable, and even more valuable, in
the design of new protocols (as we illustrated elsewhere, for a MAC
protocol for the SWIFT optically switched network).  For TCP and
Sockets, our specifications had to capture the historical
complexities, whereas for a new protocol design, such specification
and testing can identify unintended complexities at an early point in
the design.
